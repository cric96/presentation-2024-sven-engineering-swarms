%%%%%%%%%%%%%%%%%%%%%%%%%%%%%%%%%%%%%%%%%%%%%%%%%%%%%%%%%%%%%%%%%%%%%%%%%%%%%%%%
% AMS Beamer series / Bologna FC / Template
% Andrea Omicini
% Alma Mater Studiorum - Università di Bologna
% mailto:andrea.omicini@unibo.it
%%%%%%%%%%%%%%%%%%%%%%%%%%%%%%%%%%%%%%%%%%%%%%%%%%%%%%%%%%%%%%%%%%%%%%%%%%%%%%%%
%\documentclass[handout]{beamer}\mode<handout>{\usetheme{default}}
%
\documentclass[presentation, 9pt]{beamer}\mode<presentation>{\usetheme{AMSBolognaFC}}
%\documentclass[handout]{beamer}\mode<handout>{\usetheme{AMSBolognaFC}}
%%%%%%%%%%%%%%%%%%%%%%%%%%%%%%%%%%%%%%%%%%%%%%%%%%%%%%%%%%%%%%%%%%%%%%%%%%%%%%%%
\usepackage[T1]{fontenc}
\usepackage{wasysym}
\usepackage{amsmath,blkarray}
\usepackage{soul}
\usepackage[minted,most]{tcolorbox}
\usepackage{centernot}
\usepackage{fontawesome}
\usepackage{fancyvrb}
\usepackage{hyperref}
\usepackage{multicol}
\setminted[scala]{fontsize=\small,frame=lines,baselinestretch=1,obeytabs=true, tabsize=2}
\setminted[yaml]{fontsize=\large,frame=lines,linenos,baselinestretch=1,obeytabs=true, tabsize=2}
\usepackage[ddmmyyyy]{datetime}
\setminted{fontsize=\footnotesize}
\renewcommand{\dateseparator}{}
%\renewcommand{\thefootnote}{\fnsymbol{footnote}}
\newcommand{\version}{1}

\usepackage[
	%backend=biber,
	backend=bibtex,
%	citestyle=authoryear-icomp,
%	maxcitenames=1,
	bibstyle=numeric,
	style=ieee]{biblatex}
\usepackage{graphicx}

	\makeatletter

%\addbibresource{biblio.bib}

\bibliography{biblio}


\newcommand\extrafootertext[1]{%
    \bgroup
    \renewcommand\thefootnote{\fnsymbol{footnote}}%
    \renewcommand\thempfootnote{\fnsymbol{mpfootnote}}%
    \footnotetext[0]{#1}%
    \egroup
}

\newcommand{\citeinslide}[1]{\cite{#1}\extrafootertext{\scriptsize\cite{#1} \fullcite{#1}}}

%%%%%%%%%%%%%%%%%%%%%%%%%%%%%%%%%%%%%%%%%%%%%%%%%%%%%%%%%%%%%%%%%%%%%%%%%%%%%%%%
\title[Engineering Swarm Behaviours through HAC]
{Engineering Swarm Behaviours through (Hybrid) Aggregate Computing}
%
%
\author[\sspeaker{Aguzzi}]
{\speaker{Gianluca Aguzzi} \href{mailto:gianluca.aguzzi@unibo.it}{gianluca.aguzzi@unibo.it} }
\institute[DISI, Univ.\ Bologna]
{%Dipartimento di Informatica -- Scienza e Ingegneria (DISI)\\
\textsc{Alma Mater Studiorum} -- Universit{\`a} di Bologna \\[0.1cm]
%\textbf{Talk @} \bold{International Conference on Autonomic Computing and Self-Organizing Systems (ACSOS)}\\[0.15cm]
%\includegraphics[width=0.15\textwidth]{img/qr-code-today.png}
}
%
\renewcommand{\dateseparator}{/}
\date[\today]{\today}
%

%%%%%%%%%%%%%%%%%%%%%%%%%%%%%%%%%%%%%%%%%%%%%%%%%%%%%%%%%%%%%%%%%%%%%%%%%%%%%%%%
\lstdefinelanguage{scala}{
  keywords={abstract,case,catch,class,def,%
    do,else,extends,false,final,finally,%
    for,if,implicit,import,match,mixin,%
    new,null,object,override,package,%
    private,protected,requires,return,sealed,%
    super,this,throw,trait,true,try,lazy,%
    type,val,var,while,with,yield,forSome},
  otherkeywords={=>,<-,<\%,<:,>:,\#},
  sensitive=true,
  morecomment=[l]{//},
  morecomment=[n]{/*}{*/},
  morestring=[b]",
  morestring=[b]',
  morestring=[b]""",
  basicstyle=\lst@ifdisplaystyle\footnotesize\fi\ttfamily,
  emphstyle=\bfseries
}
\definecolor{ddarkgreen}{rgb}{0,0.5,0}
\lstdefinelanguage{scafi}{frame=single,basewidth=0.5em,language={scala},
keywordstyle=\color{blue}\textbf, commentstyle=\color{ddarkgreen},
keywordstyle=[2]\color{red}\textbf, keywords=[2]{rep,nbr,foldhood,foldhoodPlus,aggregate,branch,spawn},
keywordstyle=[3]\color{gray}, keywords=[3]{Me,AroundMe,Everywhere,Forever}, %,@@,@@@
keywordstyle=[4]\color{red}\textbf, keywords=[4]{in,out,rd},
keywordstyle=[5]\color{violet}, keywords=[5]{evolve,when,andNext,workflow,C,gossip},
keywordstyle=[6]\color{orange}, keywords=[6]{Available,Serving,Done,Waiting,Removing}}

\newcommand{\hsplit}[2]{
\begin{minipage}{0.48\textwidth}
#1
\end{minipage}
\hfill
\begin{minipage}{0.48\textwidth}
#2
\end{minipage}
}
\newcommand{\hsplits}[4]{
\begin{minipage}{#1\textwidth}
#3
\end{minipage}
\hfill
\begin{minipage}{#2\textwidth}
#4
\end{minipage}
}

\newcommand{\lbl}[1]{\textbf{\textcolor{gray!90!white}{#1}}}
\newcommand{\enf}[1]{{\textcolor{red}{#1}}}
\newcommand{\bo}[1]{\textbf{#1}}

\newcommand{\imgh}[2]{
\begin{figure}
\centering
\includegraphics[width=#1\textwidth]{img/#2}
\end{figure}
}
\newcommand{\imgv}[2]{
\begin{figure}
\centering
\includegraphics[height=#1\textheight]{img/#2}
\end{figure}
}

\newtcblisting{mycode}[3]{%
  boxsep=0pt,
  boxrule=0pt,
  arc=1mm, 
  left=1mm,
  %auto outer arc,
  size=fbox,%tight,
  %colframe=blue!40!black, colframe=black!30!white,
  %colbacktitle=blue!80!white,
  colback=blue!5,
  %toprule=0.1mm, bottomrule=0.1mm, rightrule=0.1mm, leftrule=1mm, 
  listing only,
  listing options={language=scafi, alsoletter={-},
    backgroundcolor={},
  	columns=fullflexible,
  	lineskip={-1.5pt},
  	xleftmargin=0px,
  	belowskip={0px},
  	aboveskip={0px},
  	frame=none,
  	#2
  },
  title={#3},#1
}
\lstdefinestyle{s}{basicstyle=\ttfamily\footnotesize}
\lstdefinestyle{ss}{basicstyle=\ttfamily\scriptsize}
\lstdefinestyle{sss}{basicstyle=\ttfamily\tiny}
\lstdefinestyle{conf}{language={},morecomment=[l][\color{darkgreen}]{\#},
basicstyle=\ttfamily\scriptsize}

\newcommand{\question}[1]{\textcolor{darkgray}{\emph{\bo{#1}}}}
\newcommand{\refslide}[1]{Slide~\ref{#1}}

\begin{document}
%%%%%%%%%%%%%%%%%%%%%%%%%%%%%%%%%%%%%%%%%%%%%%%%%%%%%%%%%%%%%%%%%%%%%%%%%%%%%%%%
\input{macro-ac.tex}
\frame{\titlepage}
\begin{frame}{Me}
	\begin{columns}
		\begin{column}{0.5\textwidth}
		\centering
		\fbox{\includegraphics[width=0.5\linewidth]{img/me.jpg}}
		\\
		\vspace{0.2cm}
		\href{https://github.com/cric96}{\faGithub} \,
		\href{https://stackoverflow.com/users/10295847/gianluca-aguzzi}{\faStackOverflow} \,
		\href{https://www.linkedin.com/in/gianluca-aguzzi-265998170/}{\faLinkedin} \,
		\href{https://www.unibo.it/sitoweb/gianluca.aguzzi}{\faGlobe} \,
		\end{column}
		\begin{column}{0.5\textwidth}
			\begin{itemize}
				\item PhD Candidate in Computer Science and Engineering
				\item Research interests:
				\begin{itemize}
					\item Multi-agent systems
					\item Distributed Collective Intelligence
					\item Deep Reinforcement Learning
					\item Multi-agent Reinforcement Learning
					\item Distributed Macroprogramming
				\end{itemize}
				\item One of the lead developer of \href{https://scafi.github.io/}{ScaFi}
				%\item Scala Lover \& Functional Programming enthusiast
			\end{itemize}
		\end{column}
	\end{columns}
\end{frame}
\begin{frame}
\frametitle{Outline}
\tableofcontents
\end{frame}
\section{Swarm Behaviors -- Overview}
\begin{frame}{Swarm Behaviors}
\begin{alertblock}{Definition}
	A \emph{swarm behavior} a \underline{collective} behavior that emerges from the interaction of a group of agents. 
\end{alertblock}
\begin{exampleblock}{Taxonomy}
	From a swarm robotics perspective, swarm behaviors can be classified into three main categories~\citeinslide{DBLP:journals/swarm/BrambillaFBD13}:
	\begin{itemize}
		\item \emph{Spatial-organization behaviors}: the agents organize themselves in a specific spatial configuration
		\item \emph{Navigation behaviors}: the agents collectively navigate in the environment following a specific strategy
		\item \emph{Collective Decision-Making}: the agents collectively make decisions based on the information they have
	\end{itemize}
\end{exampleblock}
\end{frame}
\begin{comment}
\begin{frame}{Spatial-organization behaviors}
\begin{center}
	\begin{columns}
		\begin{column}{0.31\textwidth}
			\includegraphics[width=\textwidth]{example-image-a}
			\centering
			\Large{\textbf{Aggregation}}
		\end{column}
		\begin{column}{0.31\textwidth}
			\includegraphics[width=\textwidth]{example-image-a}
			\centering
			\Large{\textbf{Dispersion}}
		\end{column}
		\begin{column}{0.31\textwidth}
			\includegraphics[width=\textwidth]{example-image-a}
			\centering
			\Large{\textbf{Flocking}}
		\end{column}
	\end{columns}
\end{center}
\end{frame}
\begin{frame}{Navigation behaviors}
\begin{center}
	\begin{columns}
		\begin{column}{0.32\textwidth}
			\includegraphics[width=\textwidth]{example-image-a}
			\centering
			\Large{\textbf{Exploration}}
		\end{column}
		\begin{column}{0.32\textwidth}
			\includegraphics[width=\textwidth]{example-image-a}
			\centering
			\Large{\textbf{Collective Motion}}
		\end{column}
		\begin{column}{0.32\textwidth}
			\includegraphics[width=\textwidth]{example-image-a}
			\centering
			\Large{\textbf{Group Transportation}}
		\end{column}
	\end{columns}
\end{center}
\end{frame}
\begin{frame}{Collective Decision-Making}
	\begin{columns}
		\begin{column}{0.32\textwidth}
			\includegraphics[width=\textwidth]{example-image-a}
			\centering
			\Large{\textbf{Exploration}}
		\end{column}
		\begin{column}{0.32\textwidth}
			\includegraphics[width=\textwidth]{example-image-a}
			\centering
			\Large{\textbf{Collective Motion}}
		\end{column}
	\end{columns}
\end{frame}
\end{comment}
\begin{frame}[allowframebreaks]{Engineering Swarm Behaviors}
	\begin{alertblock}{Definition}
		\emph{Engineering swarm behaviors} is the process of \underline{designing}, \underline{implementing}, and \underline{deploying} algorithms that enable a group of agents to exhibit a specific collective behavior.
	\end{alertblock}
	\begin{exampleblock}{Challenges}
		\begin{itemize}
			\item \emph{Scalability}: the collective behavior should scale with the number of agents
			\item \emph{Robustness}: even in case of failures, the collective behavior should be maintained
			\item \emph{Flexibility}: the collective behavior should be easily adaptable to different scenarios
			\item \emph{Efficiency}: the collective behavior should be efficient in terms of energy and time
		\end{itemize}
	\end{exampleblock}
	\begin{exampleblock}{Properties}
		\begin{itemize}
			\item \emph{Emergence}: the collective behavior emerges from the interaction of the agents
			\item \emph{Decentralization}: the collective behavior is achieved \underline{without} a central controller
			\item \emph{Self-organization}: the collective behavior is achieved through \underline{local} interactions
			\item \emph{Homogeneity}: the agents are \underline{homogeneous} and have the same capabilities
			\item \emph{Limited Communication}: the agents have limited communication capabilities
		\end{itemize}
	\end{exampleblock}
	\begin{block}{State-of-the-art}
		\begin{itemize}
			\item \emph{manual design}: the collective behavior is manually designed through 
			\begin{itemize}
				\item bio-inspired algorithms, macroprogramming languages \faArrowRight \, \underline{aggregate programming}
			\end{itemize}
			\item \emph{automatic design}: the collective behavior is automatically designed through machine learning techniques
			\begin{itemize}
				\item reinforcement learning, evolutionary algorithms, \underline{multi-agent reinforcement learning}
			\end{itemize}
			\item \emph{\bold{hybrid design}}: a combination of manual and automatic design
		\end{itemize}
	\end{block}
\end{frame}
\section{Aggregate Computing Toolchain}
\subsection{ScaFi}
\subsection{MacroSwarm}
\section{Hybrid Aggregate Computing}
\subsection{Execution strategy}
\subsection{Application level}

%===============================================================================
\begin{frame}{Aggregate Programming -- What we want}
	\begin{exampleblock}{Goal}
		\begin{itemize}
			\item a \bold{minimal core} to cover the main functionality of field calculus
			\item a \bold{set of libraries} that cover the main building blocks
			\item an \bold{high-level API} for domain-specific applications
		\end{itemize}
	\end{exampleblock}
	\begin{exampleblock}{Challenges}
		\begin{itemize}
			\item Support several kinds of devices
			\begin{itemize}
				\item \emph{constrained, commodity, consumer, server}
			\end{itemize}
			\item Support several infrastructures
			\begin{itemize}
				\item \emph{peer-to-peer, server-based, cloud-edge} %-- \underline{pulverization? Nicolas talk}
			\end{itemize}
			\item Expressive and simple enough to reduce the cognitive load
		\end{itemize}
	\end{exampleblock}
\end{frame}

\begin{frame}[allowframebreaks]{Aggregate programming: current implementations}
	\begin{center}
	A large number of incarnations and implementations have been developed over the years, with different \emph{trade-offs} and \emph{goals}.
	\end{center}
	\begin{exampleblock}{External DSL}
		\begin{itemize}
			\item \bold{Proto}~\cite{proto}: first incarnation of FC
			\begin{itemize}
				\item \faThumbsUp \, \emph{good performance} \faArrowRight \, \emph{low-level}
				\item \faThumbsDown \, \emph{hard to use} \faArrowRight \, \emph{hard to maintain}
			\end{itemize}
			\item \bold{Protelis}~\cite{protelis}: an external DSL based on Xtext
			\begin{itemize}
				\item \faThumbsUp \, \emph{good ergonomics} \faArrowRight \, \emph{high-level}
				\item \faThumbsUp \, \emph{extensive libraries} \faArrowRight \, \emph{easy to use}
				\item \faThumbsDown \, \emph{hard to maintain}
				\item \faThumbsDown \, \emph{JVM-based} \faArrowRight \, \emph{hard to port to constrained devices}
			\end{itemize}
		\end{itemize}
	\end{exampleblock}

	\begin{exampleblock}{Internal DSL}
		\begin{itemize}
			\item \bold{FCPP}~\cite{fcpp}: a performance-oriented internal DSL in C++
			\begin{itemize}
				\item \faThumbsUp \, \emph{good performance} \faArrowRight \, \emph{low-level}
				\item \faThumbsDown \, \emph{hard to use} \faArrowRight \, \emph{bad ergonomics}
			\end{itemize}
			\item \underline{\bold{ScaFi}}~\cite{scafi}: a high-level internal DSL in Scala (current reference implementation and focus of this talk)
			\begin{itemize}
				\item \faThumbsUp \, \emph{good ergonomics} \faArrowRight \, \emph{high-level}
				\item \faThumbsUp \, \emph{extensive libraries} \faArrowRight \, \emph{easy to use}
				\item \faThumbsUp \, \emph{Cross-platform}
				\item \faThumbsDown \, Scala Native \faArrowRight \, \emph{hard to port to constrained devices}
			\end{itemize}
		\end{itemize}
	\end{exampleblock}

    \begin{exampleblock}{Experimental}
			\begin{itemize}
        \item \bold{Collektive}\footnote{https://github.com/Collektive/collektive}: a high-level DSL in Kotlin leveraging the compiler plugin and the Kotlin Multiplatform
        \item \bold{RuFi}\footnote{https://github.com/lm98/rufi}: a minimal core in Rust (WIP)
			\end{itemize}
		\end{exampleblock}
\end{frame}
\begin{frame}[c, plain]
\begin{center}
	%\includegraphics[width=0.2\textwidth]{img/qr-code-scafi.png}\\
	{\Huge \textbf{ScaFi} (\bold{Sca}la \bold{Fi}elds)}\\
	{\large A Scala toolkit providing an \emph{internal domain-specific} language, \emph{libraries}, a \emph{simulation} environment, and \emph{runtime} support for \bold{practical} aggregate computing systems development} \\[0.3cm]
	\includegraphics[width=0.2\textwidth]{img/gradient-scafi.png}
	\includegraphics[width=0.35\textwidth]{img/scr-result.png}
	\includegraphics[width=0.215\textwidth]{img/obstacle-avoidance.png}
\end{center}
\end{frame}
\begin{frame}{ScaFi: Organization}
\centering
\includegraphics[width=0.9\textwidth]{img/scafi-project-org.pdf}
\end{frame}
%\begin{frame}{ScaFi: Design}
%\centering
%\includegraphics[width=0.8\textwidth]{img/scafi-design.drawio.pdf}
%\end{frame}

\begin{frame}[fragile]{ScaFi: syntax as a core language/API}

	\begin{mycode}{}{}{}
trait Constructs {
	// the unique identifier of the local device 
	def mid(): ID
	
	// applies fun to the previous result, or to init at the first call
	def rep[A](init: A)(fun: (A) => A): A
	
	// evaluation of expr at the currently-considered neighbor
	def nbr[A](expr: => A): A
	
	// accumulates available evaluations of expr, with acc/init monoid
	def foldhood[A](init: => A)(acc: (A,A)=>A)(expr: => A): A
	
	// splits computation: th where cond is true, el everywhere/time else
	def branch[A](cond: => Boolean)(th: => A)(el: => A): A
	
	// perception of local sensor
	def sense[A](name: CNAME): A
	
	// perception of neighbourhood sensor
	def nbrvar[A](name: CNAME): A
	...
}
\end{mycode}
\end{frame}
\begin{frame}{ScaFi: Cross-platform}
	%ScaFi leverages Scala.js and Scala Native to be a multi-target framework
	%\begin{itemize}
	%	\item ScaFi-web~\cite{DBLP:conf/coordination/AguzziCMPV21} is an example of an application that uses the js variant of ScaFi.
	%\end{itemize}
	\centering
	\includegraphics[width=0.9\textwidth]{img/multi-platform.drawio}
\end{frame}
\begin{frame}[fragile]{ScaFi: setup an aggregate application}
\begin{mycode}{}{}{}
package experiments
// STEP 1 Choose an incarnation (simulated or real)
import it.unibo.scafi.incarnations.BasicSimulationIncarnation._

// STEP 2 Define the aggregate program including the right libraries
class MyProgram extends AggregateProgram with Libs {
	// STEP 2.1 Define main logic of the program
	override def main(): Any = rep(0)(_ + 1)
}

// STEP 3 Platform/Node setup (both in simulation/real)

// STEP 3.1 in case of ScaFi simulation:
object SimulationRunner extends Launcher {
  Settings.Sim_ProgramClass = "experiments.MyAggregateProgram"
  Settings.ShowConfigPanel = true
  launch()
}
\end{mycode}
\end{frame}
\begin{frame}{An aggregate computing playground -- ScaFi Web!}
\centering
\url{https://scafi.github.io/web/}
\centering
\includegraphics[width=0.8\textwidth]{img/gradient-web.png}
\end{frame}

\begin{frame}{ScaFi Blocks}
ScaFi Blocks is a visual programming environment for ScaFi.
\centering
\includegraphics[width=\textwidth]{img/scafiblocks.png}
\end{frame}
\begin{frame}{MacroSwarm}
	\begin{alertblock}{What}
		MacroSwarm\footnote{\url{https://scafi.github.io/macro-swarm/}}~\cite{macroswarm-arxiv} is a ScaFi library for programming swarm robotics applications.
	\end{alertblock}
	\begin{exampleblock}{How}
		\begin{itemize}
			\item A MacroSwarm program is a function from \emph{sensing} field to \emph{action} field
			\begin{itemize}
				\item \emph{sensing} field: the information perceived by the agents (e.g., the distance from a source)
				\item \emph{action} field: the action to be performed by the agents (e.g., a velocity vector)
			\end{itemize}
			\item The action field then should be translated into the actual action of the agents based on the specific robot capabilities
			\begin{itemize}
				\item e.g., a velocity vector can be translated into motor commands
			\end{itemize}
		\end{itemize}
	\end{exampleblock}
	%\centering
	%
\end{frame}
\begin{frame}{MacroSwarm: API structure}
	\centering
	\includegraphics[width=\textwidth]{img/api-structure.drawio.pdf}
\end{frame}
\begin{frame}{MacroSwarm: Example}
	\centering
	\includegraphics[width=0.9\textwidth]{img/macro-swarm.png}
\end{frame}

\begin{frame}{Towards Hybrid Aggregate Computing}
	\begin{alertblock}{What}
		\emph{Hybrid Aggregate Computing} is the process of combining \emph{manual} (aggregate programming) and \emph{automatic} design techniques to engineer swarm behaviors.
	\end{alertblock}
	\begin{exampleblock}{Why}
		\begin{itemize}
			\item \emph{Manual design} is good for designing \emph{simple} and \emph{well-understood} behaviors
			\item \emph{Automatic design} is good for designing \emph{complex} and \emph{unknown} behaviors
			\item \emph{Hybrid design} is good for combining the best of both worlds
		\end{itemize}
	\end{exampleblock}
\end{frame}

\begin{frame}{Research Agenda~\citeinslide{DBLP:conf/icdcs/AguzziCV22}}
	\bold{Goals}:
	\begin{itemize}
	  \item \underline{functionality} (e.g., a certain collective behaviour)
	  \item \underline{non-functionality} (e.g., energy consumption)
	\end{itemize}
	\bold{Means}:
	\begin{itemize}
	  \item \underline{system structures}
	  \item \underline{\emph{execution strategy}}
	  \item \underline{\emph{algorithms}}
	\end{itemize}
	\begin{center}
	  \includegraphics[width=0.6\textwidth]{img/roadmap.pdf}
	\end{center}
\end{frame}
\begin{frame}{Hybrid Aggregate Computing -- Execution Strategy}
	\begin{exampleblock}{Distributed schedulers}
	  \begin{itemize}
		\item The aggregete computing model does not \underline{enforce} a global-synchronization
		\item \emph{\underline{Distributed}} schedulers~\citeinslide{DBLP:conf/acsos/AguzziCV22}: \emph{\underline{local}} decisions on the \emph{\underline{rounds}} of the aggregate program
		\item Idea: \emph{\underline{learning}} the \emph{\underline{best}} scheduling strategy directly by-doing
	  \end{itemize}
	  \begin{center}
	  \includegraphics[width=0.9\textwidth]{img/algorithm-learning.png}
	  \end{center}
	\end{exampleblock}
\end{frame}
\begin{frame}{Hybrid Aggregate computing -- Algorithms}
	\begin{exampleblock}{Collective Program Sketching}
	  \begin{itemize}
		\item \emph{\underline{holes}}: \emph{\underline{blocks}} of the aggregate program depending on \emph{\underline{environment dynamics}}
		\item \emph{\underline{sketching}}: \emph{\underline{partial}} specification of the aggregate program
		\item Idea~\citeinslide{DBLP:conf/coordination/AguzziCV22}: \emph{\underline{synthesis}} of the \emph{\underline{holes}} by \emph{\underline{learning}} from the \emph{\underline{realistic simulation}}
	  \end{itemize}
	  \begin{center}
	  \includegraphics[width=0.5\textwidth]{img/synthesis-3.png}
	  \end{center}
	\end{exampleblock}
\end{frame}
\begin{frame}{Hybrid Aggregate computing -- Algorithms}
	\begin{exampleblock}{Field-Informed reinforcement learning}
	  \begin{itemize}
		\item Aggregate computing is used to \emph{\underline{inform}} the learning process
		\begin{itemize}
		  \item Speeding up the training \& Reducing the search space \citeinslide{acgnn}
		\end{itemize} 
		\item Field used as a way to produce a stygmergic representation of the environment
		\item Learning performed through \emph{\underline{deep reinforcement learning}}
		\item Graph-neural networks used as a policy function approximator
	  \end{itemize}
	\centering
	  \includegraphics[width=0.6\textwidth]{img/architecture.pdf}
	\end{exampleblock}
\end{frame}
\begin{frame}{Conclusion}
	\begin{alertblock}{Takeaways}
		\begin{itemize}
			\item \emph{Swarm behaviors} are collective behaviors that emerge from the interaction of a group of agents
			\item \emph{Aggregate computing} is a programming model that provides a high-level API for engineering swarm behaviors
			\item \emph{ScaFi} is a Scala toolkit that provides an internal DSL, libraries, a simulation environment, and runtime support for practical aggregate computing systems development
			\item \emph{MacroSwarm} is a ScaFi library for programming swarm robotics applications
			\item \emph{Hybrid Aggregate Computing} is the process of combining manual and automatic design techniques to engineer swarm behaviors
		\end{itemize}
	\end{alertblock}
\end{frame}
%===============================================================================

%/////////
\frame{\titlepage}
%/////////

%===============================================================================
\section*{\refname}
%===============================================================================

%%%%
\setbeamertemplate{page number in head/foot}{}
%/////////
\begin{frame}[allowframebreaks]{References}
\def\bibfont{\footnotesize}
\printbibliography
\end{frame}

%%%%%%%%%%%%%%%%%%%%%%%%%%%%%%%%%%%%%%%%%%%%%%%%%%%%%%%%%%%%%%%%%%%%%%%%%%%%%%%%
\end{document}
%%%%%%%%%%%%%%%%%%%%%%%%%%%%%%%%%%%%%%%%%%%%%%%%%%%%%%%%%%%%%%%%%%%%%%%%%%%%%%%%
